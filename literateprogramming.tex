% literateprogramming.tex
%
% Folien zu Fallstuden-Vortrag über das Paper
% 	D.E. Knuth
% 	Literate Programming
% 	The Computer J. 27 (1984) 97-111
% 
% Raphael Das Gupta <raphaeld@student.ethz.ch>
% (K) 2007 -- ALL RITES REVERSED -- Reprint what you like.
%
% Quellen:
%	* Das Paper selbst:
%		http://www.literateprogramming.com/knuthweb.pdf
%
\documentclass[10pt]{beamer}
%
\usepackage[utf8]{inputenc}
\usepackage[all]{xy}
\usepackage{verbatim}

\def\WEB{{\tt WEB}}
\def\PASCAL{{\rm PASCAL}}
\def\C{{\rm C}}
\def\Cpp{{\rm C++}}

\mode<beamer>{%
	\usetheme[hideothersubsections,
	right, width=22mm]{Goettingen}
}

\title{Donald E. Knuth: Literate Programming}
\author[R. Das Gupta]{Raphael Das Gupta}
\institute{ETH Zürich}
%\titlegraphic{\includegraphics[width=20mm]{test}}
\date{25.01.2007}

\begin{document}
%%%%    T I T E L S E I T E     %%%%
\begin{frame}<handout:0>  % nicht im Handout
	The Computer J. 27 (1984) 97-111
	\titlepage
\end{frame}

\section{Die Idee}

\begin{comment}
\begin{frame}
	\frametitle{Programme als literarische Werke}\pause
	\begin{block}{``Was soll das nun heissen?''}
	\end{block}
\end{frame}

\subsection{Vorraussetzungen}

\begin{frame}
	\frametitle{Vorraussetzungen}
	\begin{itemize}
		\item Strukturierte Programme\pause
		\item Bedürfnis nach besserer Dokumentation
	\end{itemize}
\end{frame}

\subsection{Ziel}
\end{comment}

\begin{frame}
	\frametitle{Die Idee: Programme als literarische Werke}
	\begin{block}{Änderung der Grundeinstellung gegenüber dem, was ein Programm ist/soll}
		Hauptzweck eines Programms:
		\begin{itemize}
			\item nicht mehr den Computer anzuweisen, was er zu tun hat \pause \dots
			\item sondern einem Menschen erklären was der Computer tun soll
		\end{itemize}
	\end{block}
\end{frame}
\begin{comment}
\begin{frame}
	\frametitle{Die Idee: Programme als literarische Werke}
	{Das Programm muss:}
	\begin{itemize}
		\item ``angenehm'' zu lesen und \dots
		\item gut verständlich sein\pause
	\end{itemize}
		{Es soll trotzdem:}
	\begin{itemize}
		\item korrekt kompilieren
		\item dann das tun, was es sollte
	\end{itemize}
\end{frame}
\end{comment}

\begin{frame}
	\frametitle{Angenehmer Nebeneffekt}
	Zwang, die Motivation zu Hinterfragen\pause \\
	$\Rightarrow$ Durchdachtere Programme
\end{frame}

\section{\WEB}
\begin{frame}
	\frametitle{\WEB}
	Komplexe Programme als
	\begin{itemize}
		\item einfache (kleine) Programmteile\pause und
		\item einfache Beziehungen zwischen diesen
	\end{itemize}
\end{frame}
\subsection{Prinzip}

\begin{frame}
	\frametitle{Funktionsprinzip von \WEB}
	\begin{block}{Zweisprachigkeit}
		\begin{itemize}
			\item Formatierungssprache: \pause \TeX
			\item Programmiersprache: \pause \PASCAL
		\end{itemize}
	\end{block}
\end{frame}

\begin{frame}
	\frametitle{Zwei Toolchains}
	\begin{displaymath}
		\xymatrix{ & \tt TEX \ar[r]^{\rm{TeX}} & \tt PDF/DVI \\
		          \tt WEB \ar[ru]^{\tt{weave}} \ar[rd]_{\tt{tangle}}  &         &             \\
		           & \tt PAS \ar[r]_\PASCAL & \tt BINARY}
	\end{displaymath}
\end{frame}

\subsection{weben, {\TeX}en, lesen}

\begin{frame}
	\frametitle{Aufbau der erzeugten Dokumentation}
	\begin{itemize}
		\uncover<6->{\item Inhalt}
		\item Kapitel
		\uncover<2->{
		\begin{itemize}
			\item{Titel}
			\item{Beschreibung} \uncover<3->{: \TeX}
			\item{Substitutionen}\uncover<4->{, z.B.:\\
				{\tt {\textbf{define} print\_string(\#)$\equiv$write(\#)}}}
			\item{Quelltext}\uncover<5->{: Code-Blöcke}
		\end{itemize}
		}
		\uncover<7->{\item Index}
	\end{itemize}
\end{frame}


\subsection{verwickeln, kompilieren, laufenlassen}

\begin{frame}
	\frametitle{verwickeln, kompilieren, laufenlassen}
	Quelltext ``unlesbar''
\end{frame}

\subsection{Features}

\begin{frame}
	\begin{itemize}
		\item Spezielle Behandlung von {\tt char}s und {\tt string}s
		\item changefiles
		\item kein Zwang zu top-down oder bottom-up
		\item simple Arithmetik zur teangle-Zeit
	\end{itemize}
\end{frame}

\section{Ausblick}

\begin{frame}
	\frametitle{Andere Literate-Programming-Tools}
	\begin{itemize}
		\item cweb
		\uncover<2->{
			\begin{itemize}
				\item \C, \Cpp, (java)
				\item \TeX
			\end{itemize}
		}
		\item noweb		
		\uncover<3->{
			\begin{itemize}
				\item beliebige Programmiersprachen
				\item \TeX, \LaTeX, html, troff
			\end{itemize}
		}
		\item funnelweb
		\uncover<4->{
			\begin{itemize}
				\item beliebige Programmiersprachen
				\item \TeX, html
			\end{itemize}
		}
		\item LEO ``Literate Editor with Outlines''
	\end{itemize}
	\uncover<5->{
		inzwischen auch interaktive Dokumentationen möglich
	}
\end{frame}

\begin{frame}
	\frametitle{Semi-Literate Programming}
	z.B. javadoc, doc++

	Hauptunterschied: bei ``Full-Literate'' kann Reihenfolge der Kapitel frei gewählt werden
\end{frame}

\section{}
\begin{frame}
	http://literateprogramming.com/
\end{frame}

\end{document}

